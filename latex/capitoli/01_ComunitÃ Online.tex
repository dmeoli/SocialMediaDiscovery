\chapter{Comunit{\`a} Online}

Le comunit{\`a} online sono una realt{\`a} collaborativa che trovano spazio in molteplici ambiti: dallo sviluppo di progetti software in comunit{\`a} aperte (es. MySQL developer community) o chiuse, all'interno di organizzazioni aziendali (distribuite sul territorio), alla ricerca di lavoro e pubblicazione di profili professionali (es. LinkedIn), alla condivisione di informazioni di singoli individui (es. Facebook). Le comunit{\`a} online, in quanto tali, agevolano collaborazioni e scambio di conoscenza, abbattendo spesso barriere organizzative e geografiche. Molte di esse interagiscono tramite strumenti di comunicazione e dispongono anche di soluzioni di data storage che memorizzano informazioni di vario genere, relative alla comunit{\`a} medesima. 

\section{Obiettivi}
Con la finalit{\`a} di modellare comunit{\`a} collaborative attraverso strumenti analitici, si considerino gli specifici sotto-obiettivi scientifico-tecnologici di seguito elencati:

\begin{enumerate}[label=\theenumi.]

\item \textit{Identificazione e selezione di sorgenti dati prodotti da comunit{\`a}}: il recente paradigma di data science suggerisce di investigare sorgenti dati da cui estrarre informazioni utili per la modellazione computazionale della comunit{\`a}. In tal senso, saranno considerate sorgenti di dati non strutturate prodotte da piattaforme tecnologiche che supportano comunit{\`a} digitali. Una strategia data-driven {\`e} innovativa rispetto all'usuale approccio basato sull'intervento di esperti con forte background sociologico. L'innovativit{\`a} diventa pi{\`u} significativa dal momento che l'analisi considerer{\`a} osservazioni puntuali della comunit{\`a}, nella forma di comunicazioni e messaggi prodotte/consumate al suo interno;

\item \textit{Modellazione della comunit{\`a} con strumenti di analisi}: considerando che una comunit{\`a} {\`e} un dominio complesso caratterizzato da partecipanti, relazioni ed interazioni tra questi e ruoli da loro ricoperti, una prima problematica da investigare {\`e} lo studio di modelli computazionali adeguati a rappresentare le diverse componenti di una comunit{\`a}. In questo senso si investigheranno soluzioni per l'estrazione di informazioni che caratterizzano una comunit{\`a} dalle sorgenti precedentemente identificate e selezionate. In questo senso, l'innovazione consister{\`a} nell'uso di strumenti di elaborazione del linguaggio naturale per analizzare comunicazioni e messaggi al fine di individuare informazioni che caratterizzano i singoli individui e le loro interazioni. Ci si propone di progettare ed implementare strumenti prototipali in grado di sfruttare le informazioni precedentemente estratte per fornire un modello computazionale della comunit{\`a}. L'innovazione risiede nel problema e nella soluzione computazionale. Si ritiene innovativo analizzare lo storico di una comunit{\`a} per trarne una caratterizzazione rispetto alla struttura, cos{\`i} come progettare strumenti di analisi dei dati per caratterizzare una comunit{\`a} attraverso la scoperta delle sue collaborazioni intrinseche e dei principali patterns di interazione soggetti a cambiamenti;

\item \textit{Sperimentazione del prototipo}: ci si propone di sperimentare il prototipo sulle sorgenti dati di partenza al fine di raccogliere specifiche computazionali quantitative attraverso consolidati protocolli di validazione empirica. Una prima innovazione sta nella definizione di specifiche quantitative che, a differenza di quelle qualitative, non risultano essere investigate per la modellazione di comunit{\`a} digitali. Una seconda strada prevede l'analisi di aspetti tecnici e tecnologici del prototipo per renderlo generale ed applicabile anche a comunit{\`a} digitali differenti da quella/e dell'obiettivo specifico. 

\end{enumerate}

\section{Risultati}

Si prevedono risultati intermedi e finali nella forma di deliverable digitali, pubblicazioni scientifiche e prototipi software organizzati attraverso i seguenti punti:

\begin{enumerate}[label=\theenumi.]

\item \label{itm:1} \textit{Sorgenti dati prodotte da comunit{\`a} collaborative}: trattasi di un report tecnico strutturato in due parti: 

\begin{enumerate}[label=(\alph*)]

\item la prima sar{\`a} incentrata sulla descrizione delle sorgenti dati, prodotte da comunit{\`a} digitali e accessibili pubblicamente, dal punto di vista della tipologia della comunit{\`a} e dal punto di vista della tipologia delle informazioni contenute;

\item la seconda parte verter{\`a} su progettazione ed implementazione di un prototipo software in grado di analizzare comunicazioni e messaggi dalle sorgenti dati, identificando informazioni di interesse per la costruzione di un modello a grafo della comunit{\`a};

\end{enumerate}

\item \textit{Scoperta di patterns di interazione}: trattasi di un report tecnico in cui viene dettagliata la progettazione ed implementazione di un prototipo software in grado di analizzare i dati in forma di grafo, al fine di scoprire conoscenza nella forma di patterns di interazione. Il risultato sar{\`a} disponibile anche in forma di prototipo software multipiattaforma open-source;

\item \textit{Validazione e applicazione dei prototipi}: trattasi di un report tecnico in cui viene descritta la sessione sperimentale volta a testare i prototipi sui dati prodotti al risultato \ref{itm:1} Il report fornir{\`a}:

\begin{enumerate}[label=(\alph*)]

\item specifiche quantitative ed interpretazioni qualitative di patterns di interazione risultanti dalla sperimentazione;

\item specifiche tecniche ed eventuali risultati sull'applicazione dei prototipi a comunit{\`a} digitali differenti da quelle di riferimento usate nel progetto;

\end{enumerate}

\item \textit{Sito web}: trattasi di un contenitore, pubblicamente accessibile, dei vari prodotti del progetto, comprensivi delle relative pubblicazioni scientifiche. 

\end{enumerate}